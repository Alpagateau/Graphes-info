Au cours de ce projet je me suis familiarisé avec la méthode du test du $\chi^2$, permettant d'évaluer l'indépendance entre des variables qualitatives, ainsi qu'avec l'analyse des correspondances multiples (ACM), qui offre un cadre puissant pour explorer les relations entre plusieurs variables qualitatives simultanément. L'application de ces méthodes au jeu de données sur les bières américaines m'a permis de découvrir des motifs intéressants, tels que les dépendances entre certaines caractéristiques géographiques et stylistiques des bières, ainsi que les principaux axes de variation dans les styles de bières produits aux États-Unis.\par
Faute de temps je n'ai pas pu entrer plus en détail dans l'établissement de profils types d'instances, ni approfondir l'interprétation des axes factoriels obtenus via l'ACM. Néanmoins, les résultats obtenus fournissent une base solide pour de futures analyses, qui pourraient inclure l'exploration de sous-groupes spécifiques de bières ou l'intégration de variables quantitatives pour enrichir l'analyse.\par
Pour finir, ce projet m'a permis, en faisant l'effort de faire le plus de calculs possible sans bibliothèque, de renforcer mes compétences en traitement de l'information, aussi bien dans la pratique que dans la théorie.