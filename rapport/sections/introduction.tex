Ce projet porte sur une application des méthodes d'analyse de données étudiées en cours de traitement de l'information. J'ai choisi dans un tel cadre d'analyser une base de données répertoriant des informations sur des bières \cite{2}. L'objectif du projet est de dégager de l'information non triviale à partir d'un tel jeu de de données. \\\par
Pour analyser ces données qualitatives, j'ai décidé dans un premier temps, de déterminer d'éventuelles indépendances entre les paramètres avec la méthode du test $\chi^2$, puis d'approfondir l'analyse en effectuant une analyse des corespondances multiples (ACM) afin de déterminer les profils typiques de bières, ainsi que ce qui les distingue. \\\par
Dans un premier temps nous aborderons brièvement le sujet du traîtement adressé au jeu de données afin de pouvoir le rendre exploitable. Nous poursuivrons en détaillant la mise en {\oe}uvre des algorithmes utilisés, pour déboucher sur l'analyse et l'interprétation des résultats.

