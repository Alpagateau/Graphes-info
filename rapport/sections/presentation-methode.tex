Dans le cadre de la construction de la table de contingence et du test $\chi^2$, le travail se fait sur chaque paire de critères possible, mais dans une démarche explicative, j'aborderai le cas de la comparaison de \code{type} avec \code{state}.

\subsection{Construction de la table de contingence}
Comme illustré dans la figure~\ref{fig:csv-plusieurs-valeurs} nos variables qualitatives sont parfois multiples, c'est-à-dire qu'à une case données du jeu de données, apparaissent plusieurs valeurs de critères. Nous allons voir un moyen de pallier à une telle spécificité pour construire la table de contingence.

\begin{figure}[H]
    \scriptsize 
    \input{figures/csv-plusieurs-valeurs}
    \caption{Trois cas de multiplicité de la variable \code{style}}

    \label{fig:csv-plusieurs-valeurs}
\end{figure}

Puisque les variables que nous manipulons sont non binaires, nous construisons comme montré en figure~\ref{fig:dummy-table}, dans le cadre de l'analyse des correspondances multiples, un tableau disjonctif complet (dit "Dummy variable table" dans le cours) pour chaque variable du jeu de données, de sorte à avoir des variables toutes binaires.

\begin{figure}[H]
    \input{figures/dummy-table}
    \caption{Tableaux disjonctifs complets de \code{style} et \code{state}}
    \label{fig:dummy-table}
\end{figure}

Ainsi, la table de contingence $X$ entre les variables \code{style} et \code{state} se détermine à partir des tableaux disjontifs complets $T_\code{style}$ et $T_\code{state}$ via la formule~:
$$X = T_\code{style}^\top \times T_\code{state}$$
de telle sorte que $X_{i,j}$ désigne le nombre de bières du $i$-ème style, originaires du $j$-ème état des États-Unis. \par
En raison de la multiplicité des variables précédemment évoquée, le tableau des probabilités se détermine non pas en divisant les coefficients de $X$ par $I$ mais par la somme des coefficients de $X$. Cependant, une instance qui aura plus de modalités (\textit{i.e.} plus de cas de multiplicités de variables) aura plus d'importance dans l'analyse qu'une instance qui n'apporte qu'une modalité par variable. Dans le cadre de la méthode de Jean-Paul Benzécri\cite{4}, nous conserverons ce phénomène en faisant l'hypothèse que les individus riches en modalités sont plus informatifs.\\\par
Pour chaque couple $(V_1, V_2)$ de variables, nous entreprenons alors une méthode du test $\chi^2$ pour déterminer s'il est possible ou non d'affirmer que $V_1$ est indépendante de $V_2$.

\subsection{Analyse des correspondances multiples}

L'analyse des correspondances multiples (ACM) est une extension de l'analyse des correspondances simples, permettant d'étudier les relations entre plusieurs variables qualitatives simultanément. L'objectif principal de l'ACM est de mettre en évidence~:
    \begin{itemize}
        \item Des axes qui polarisent la dépendance entre les différentes modalités des variables qualitatives (via l'étude des colonnes de la table de contingence) ;
        \item Des axes qui polarisent les instances selon certains critères (via l'étude des lignes de la table de contingence).
    \end{itemize}    

    L’analyse des correspondances multiples (ACM) est une extension de l’analyse factorielle des correspondances, permettant d’étudier simultanément les relations entre plusieurs variables qualitatives. L’objectif principal de l’ACM est de mettre en évidence~:
\begin{itemize}
    \item des axes factoriels qui structurent les oppositions entre les différentes modalités des variables;
    \item la projection des instances sur ces axes, permettant d’identifier des profils d’individus.
\end{itemize}

Pour ce faire, on commence par construire la table disjonctive complète $T$ des données, où chaque colonne correspond à une modalité d’une variable qualitative (toutes mises ensemble), et chaque ligne à une instance. On calcule ensuite la matrice des probabilités $P$ en divisant $T$ par la somme de ses coefficients. On calcule alors la matrice $\tilde{F}$ comme présenté dans le cours, puis la matrice $G = \tilde{F}^\top \tilde{F}$, qui caractérise les relations entre les modalités des variables (\textit{i.e.} les colonnes). C'est donc par le spectre trié dans l'ordre décroissant de cette matrice $G$ que nous allons extraire les axes factoriels les plus importants, afin de déterminer les principales oppositions entre les modalités, puis projeter les instances sur ces axes.