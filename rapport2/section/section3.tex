%
Soit $G = (V,E)$ un graphe. On souhaite d'abord montrer $1 \Rightarrow 2$ ou 
$$
  G \text{ est un arbre } \Rightarrow G \text{ est connexe et minimal}
$$

Supposons que $G$ un arbre, non minimal.
On note ${G_1}$ le graphe connexe formé en supprimant une arrète $e_1$ de $G$, reliant les sommets $v_1, v_2 \in V$.\newline
${G_1}$ étant connexe, on a l'existance d'un chemin entre $v_1$ et $v_2$, noté $w$.\newline
Or, ${G_1} = G/e_1$, donc $w \in G$. Ainsi, il existes deux chemins allant de $v_1$ a $v_2$, il y a donc un cycle. 
Cela est une contradiction avec la supposition que $G$ soit un arbre.
Ainsi, $G$ est minimal.
\newline
\newline
Soit $G = (V,E)$ un graphe. On souhaite montrer $2 \Rightarrow 3$ ou 
$$
  G \text{ est un connexe minimal } \Rightarrow |V_G| = |E_G| + 1
$$
\begin{itemize}
\item Soit le singleton $S = \{\{1\}, \emptyset\}$ On remarque de $S$ est connexe minimal, et que $|V_S| = 1 ; |E_S| = 0$, donc $S$ vérifie $|V_S| = |E_S| + 1$.
\item Soit un graphe ${G_1} = \{V_{G_1}, E_{{G_1}}\}$ connexe minimal et vérifiant $|V_{G_1}| = |E_{{G_1}}| + 1$.
  De ${G_1}$ on construit un nouveau graphe, ${G_2} = \{ V_{G_1} \cup \{x\}, E_{{G_1}} \cup \{x, v\}\}, x \notin V_{G_1}, v \in V_{G_1}$ 
  Ainsi, ${G_2}$ est connexe minimal, car connexe et acyclique, et $|V_{G_2}| = |V_{G_1}| + 1 = |E_{{G_1}}| + 2 = |E_{G_2}| + 1$
\item On par construction que tout arbre connexe minimal peut etre construit par l'induction ci-dessus, et chacun vérifie la propriété $|V| = |E| + 1$. Donc tout graphe connexe minimal vérifie cette propriété.
\end{itemize}
