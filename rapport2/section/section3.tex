\begin{enumerate}
  
\item{}

 \includegraphics[width=0.5\textwidth]{figures/Ex2Q1.png}



\item{}
Passer d'un sommet à un autre par une arête revient à changer un bit de 0 à 1 ou vice-versa donc la distance entre les sommets 0101000111 et 0110100010 est égale au nombre de bit différent soit une distance de 5.


\item{}
0101000111 →
0111000111 →
0110000111 →
0110100111 →
0110100011 →
0110100010

est un plus court chemin allant de 0101000111 à 0110100010.

\item{}
\begin{lstLNat}
    fonction distance(a: entier, b:entier) $\rightarrow$ entier 
    début
    d=0
    tant que (a != 0 et b!= 0) 
      si a\%2 != b\%2 
          d = d + 1
      fin_si
      a = a/2
      b = b/2
    fin_tant_que
    retourner d
    fin 

\end{lstLNat}
\end{enumerate}


