%
Soit $G = (V,E)$ un graphe. On souhaite d'abord montrer $1 \Rightarrow 2$ ou 
$$
  G \text{ est un arbre } \Rightarrow G \text{est connexe et minimal}
$$

Supposons que $G$ un arbre, non minimal.
On note $G_1$ le graphe connexe formé en supprimant une arrète $e_1$ de $G$, reliant les sommets $v_1, v_2 \in V$.\newline
$G_1$ étant connexe, on a l'existance d'un chemin entre $v_1$ et $v_2$, noté $w$.\newline
Or, $G_1 = G/e_1$, donc $w \in G$. Ainsi, il existes deux chemins allant de $v_1$ a $v_2$, il y a donc un cycle. 
Cela est une contradiction avec la supposition que $G$ soit un arbre.
Ainsi, $G$ est minimal.

Soit $G = (V,E)$ un graphe. On souhaite montrer $2 \Rightarrow 3$ ou 
$$
  G \text{ est un connexe minimal } \Rightarrow |V_G| = |E_G| + 1
$$

Soit le singleton $S = {{1}, \emptyset}$

