\documentclass[a4paper,12pt]{article}
\usepackage[utf8]{inputenc}
\usepackage[T1]{fontenc}
\usepackage{graphicx} % pour inclure images
\usepackage{amsmath}  % maths si nécessaire
\usepackage{amssymb}  % maths si nécessaire
% \usepackage{stmaryrd} %crochet maths
\usepackage{booktabs} % tableaux pro Pour des lignes plus jolies
\usepackage{tabularx} % Pour l'ajustement automatique de la largeur

% Language setting
\usepackage[french]{babel}
\usepackage[T1]{fontenc} 

% Set page size and margins
\usepackage[letterpaper,left=2.5cm,right=2.5cm,top=2.5cm,bottom=2.5cm,marginparwidth=1.75cm]{geometry}


% Useful packages
\usepackage{multicol} % colonnes
\usepackage[colorlinks=true, allcolors=blue]{hyperref}
\usepackage{xcolor}
\usepackage{tikz}
\usepackage{float} % pour que les figures se placent bien
\usepackage{listings} % pour inclure du code
\usepackage{caption} % pour centrer les captions malgré les \\
\captionsetup{justification=centering}


\newcommand{\N}{\mathbb{N}}
\newcommand{\code}{\texttt}
\newcommand{\mr}{\mathrm}
\renewcommand{\deg}[0]{\mathrm{deg}}
\newcommand{\card}[1]{\lvert #1 \rvert}
\newcommand{\intint}[2]{\llbracket #1 ,\, #2 \rrbracket}
\newcommand{\fonction}[5]{%
  \begin{align*}
    #1 :\ & #2 \longrightarrow #3 \\
          & #4 \longmapsto #5
  \end{align*}
}


\definecolor{keywordcolor}{RGB}{133, 153, 0}  % les mots-clés
\definecolor{commentcolor}{RGB}{147, 161, 161} % les commentaires
\definecolor{stringcolor}{RGB}{42, 161, 152}  % les chaînes de caractères

\lstnewenvironment{lstOCaml}
{\lstset{
    language=[Objective]Caml,
    basicstyle=\ttfamily,
    keywordstyle=\color{keywordcolor},
    commentstyle=\color{commentcolor},
    stringstyle=\color{stringcolor},
    backgroundcolor=\color{white},
    numbers=left,
    numberstyle=\ttfamily,
    numbersep=-1.5em,
    stepnumber=1,
    frame=l,
    framexleftmargin=-2.25em,
    tabsize=2,
    literate=%
    {é}{{\'e}}{1}%
    {è}{{\`e}}{1}%
    {à}{{\`a}}{1}%
    {ç}{{\c{c}}}{1}%
    {œ}{{\oe}}{1}%
    {ù}{{\`u}}{1}%
    {É}{{\'E}}{1}%
    {È}{{\`E}}{1}%
    {À}{{\`A}}{1}%
    {Ç}{{\c{C}}}{1}%
    {Œ}{{\OE}}{1}%
    {Ê}{{\^E}}{1}%
    {ê}{{\^e}}{1}%
    {î}{{\^i}}{1}%
    {ô}{{\^o}}{1}%
    {û}{{\^u}}{1}%
    {ä}{{\"{a}}}1
    {ë}{{\"{e}}}1
    {ï}{{\"{i}}}1
    {ö}{{\"{o}}}1
    {ü}{{\"{u}}}1
    {û}{{\^{u}}}1
    {â}{{\^{a}}}1
    {Â}{{\^{A}}}1
    {Î}{{\^{I}}}1
}}{}
 
\lstnewenvironment{lstC}
{\lstset{
    language=C,
    basicstyle=\ttfamily,
    keywordstyle=\color{keywordcolor},
    commentstyle=\color{commentcolor},
    stringstyle=\color{stringcolor},
    backgroundcolor=\color{white},
    numbers=left,
    numberstyle=\ttfamily,
    numbersep=-1.5em,
    stepnumber=1,
    frame=l,
    framexleftmargin=-2.25em,
    tabsize=2,
    literate=%
    {é}{{\'e}}{1}%
    {è}{{\`e}}{1}%
    {à}{{\`a}}{1}%
    {ç}{{\c{c}}}{1}%
    {œ}{{\oe}}{1}%
    {ù}{{\`u}}{1}%
    {É}{{\'E}}{1}%
    {È}{{\`E}}{1}%
    {À}{{\`A}}{1}%
    {Ç}{{\c{C}}}{1}%
    {Œ}{{\OE}}{1}%
    {Ê}{{\^E}}{1}%
    {ê}{{\^e}}{1}%
    {î}{{\^i}}{1}%
    {ô}{{\^o}}{1}%
    {û}{{\^u}}{1}%
    {ä}{{\"{a}}}1
    {ë}{{\"{e}}}1
    {ï}{{\"{i}}}1
    {ö}{{\"{o}}}1
    {ü}{{\"{u}}}1
    {û}{{\^{u}}}1
    {â}{{\^{a}}}1
    {Â}{{\^{A}}}1
    {Î}{{\^{I}}}1
}}{}

\lstnewenvironment{lstPython}
{\lstset{
    language=Python,
    basicstyle=\ttfamily,
    keywordstyle=\color{keywordcolor},
    commentstyle=\color{commentcolor},
    stringstyle=\color{stringcolor},
    backgroundcolor=\color{white},
    numbers=left,
    numberstyle=\ttfamily,
    numbersep=-1.5em,
    stepnumber=1,
    frame=l,
    framexleftmargin=-2.25em,
    tabsize=2,
    literate=%
    {é}{{\'e}}{1}%
    {è}{{\`e}}{1}%
    {à}{{\`a}}{1}%
    {ç}{{\c{c}}}{1}%
    {œ}{{\oe}}{1}%
    {ù}{{\`u}}{1}%
    {É}{{\'E}}{1}%
    {È}{{\`E}}{1}%
    {À}{{\`A}}{1}%
    {Ç}{{\c{C}}}{1}%
    {Œ}{{\OE}}{1}%
    {Ê}{{\^E}}{1}%
    {ê}{{\^e}}{1}%
    {î}{{\^i}}{1}%
    {ô}{{\^o}}{1}%
    {û}{{\^u}}{1}%
    {ä}{{\"{a}}}1
    {ë}{{\"{e}}}1
    {ï}{{\"{i}}}1
    {ö}{{\"{o}}}1
    {ü}{{\"{u}}}1
    {û}{{\^{u}}}1
    {â}{{\^{a}}}1
    {Â}{{\^{A}}}1
    {Î}{{\^{I}}}1
}}{}

\lstdefinelanguage{LNat}{
    morekeywords={tant,que,pour,tout,si,sinon,initialiser,renvoyer,attendre la fin, afficher, debut, fin, alors, faire, retourner, fonction},
    sensitive=false,
    morecomment=[l]{//},
}

\lstnewenvironment{lstLNat}
{\lstset{
    language=LNat,
    basicstyle=\ttfamily,
    keywordstyle=\color{keywordcolor},
    commentstyle=\color{commentcolor},
    stringstyle=\color{stringcolor},
    backgroundcolor=\color{white},
    numbers=left,
    numberstyle=\ttfamily,
    numbersep=-1.5em,
    stepnumber=1,
    frame=l,
    mathescape=true,
    framexleftmargin=-2.25em,
    tabsize=2,
    literate=%
    {é}{{\'e}}{1}%
    {è}{{\`e}}{1}%
    {à}{{\`a}}{1}%
    {ç}{{\c{c}}}{1}%
    {œ}{{\oe}}{1}%
    {ù}{{\`u}}{1}%
    {É}{{\'E}}{1}%
    {È}{{\`E}}{1}%
    {À}{{\`A}}{1}%
    {Ç}{{\c{C}}}{1}%
    {Œ}{{\OE}}{1}%
    {Ê}{{\^E}}{1}%
    {ê}{{\^e}}{1}%
    {î}{{\^i}}{1}%
    {ô}{{\^o}}{1}%
    {û}{{\^u}}{1}%
    {ä}{{\"{a}}}1
    {ë}{{\"{e}}}1
    {ï}{{\"{i}}}1
    {ö}{{\"{o}}}1
    {ü}{{\"{u}}}1
    {û}{{\^{u}}}1
    {â}{{\^{a}}}1
    {Â}{{\^{A}}}1
    {Î}{{\^{I}}}1}
}{}



\title{Rapport du TD03}
\author{Loïs Page \& Martin Nadaud \& Raphaël Jontef}
\date{S6 - Année 2025/2026}

\begin{document}

\pagenumbering{roman} % pages préliminaires


\hypersetup{pageanchor=false}
\begin{titlepage}
    \centering

    \vspace*{1.5cm}

    {\Huge\bfseries Rapport du TD03\par}

    \vspace{0.8cm}

    {\LARGE Théorie des graphes\par}

    \vspace{1.8cm}
    Loïs Page \& Martin Nadaud \& Raphaël Jontef

    \vfill

    {\large
    Semestre 6\\
    ENSEIRB-MATMECA
    \par}

    \vspace{1cm}

\end{titlepage}
\hypersetup{pageanchor=true}


%\maketitle
\tableofcontents
\newpage

\pagenumbering{arabic} % début du vrai document

\section{Exercice 1}
\begin{enumerate}
    \item \includegraphics[width=0.5\textwidth]{figures/IMG_E0894.JPG}
     \includegraphics[width=0.5\textwidth]{figures/IMG_E0895.JPG}
    \item Dans la mesure où les graphes sont représentés en machine par leur matrice d'adjacence, on peut représenter un coloriage $f$ par un tableau qui à un sommet $i$ associe sa couleur, un entier entre 1 et $k$, s'il existe $k$ couleurs.
    \item \par
\scriptsize\begin{lstPython}
    def dessiner_graphe(canvas,matrice_graphe,coordonnees_sommets,coloration=None):
        N_sommets=len(matrice_graphe)
        symetrie=(matrice_graphe==matrice_graphe.T).all()
        
        if symetrie :
            for i in range(N_sommets):
                if matrice_graphe[i,i]==1:
                    dessiner_boucle(canvas,coordonnees_sommets[i]) 
                for j in range(i+1,N_sommets):
                    if matrice_graphe[i,j]==1:
                        dessiner_arc(canvas,coordonnees_sommets[i],coordonnees_sommets[j])    
        else : 
            for i in range(N_sommets):
                for j in range(N_sommets):
                    if matrice_graphe[i,j]==1 and i!=j:
                        dessiner_arc(
                            canvas,
                            coordonnees_sommets[i],
                            coordonnees_sommets[j],
                            fleche=True
                        )
                    if matrice_graphe[i,j]==1 and i==j:
                        dessiner_boucle(canvas,coordonnees_sommets[i],fleche=True)
        if coloration != None:  
            for i in range(N_sommets):
                dessiner_sommet(canvas,coordonnees_sommets[i],str(i+1), coloration[i])    
        else:
            for i in range(N_sommets):
                dessiner_sommet(canvas,coordonnees_sommets[i],str(i+1))

        return
\end{lstPython}\normalsize
    \item Ceci revient à compter le nombre d'applications d'un ensemble de cardinal $n$ à un ensemble de cardinal $3$. Pour chacun des $n$ antécédents il y a $3$ possibilités d'images. D'où ce nombre s'élevant à $3^n$.
    \item On munit l'ensemble des applications de $[1, n]$ vers $[1, 3]$ d'un ordre total, avec comme élément minimal une certaine application $f_0$. On manipulera \code{suivant} qui à une telle application associe sa suivante selon cet ordre, où bien $f_0$ si l'antécédent est maximal.\par
\begin{lstLNat}
    est3colorable(G : graphe) $\rightarrow$ booléen:
    début
        f $\leftarrow f_0$ 
        tant que suivant(f) != $f_0$ faire
        debut
            si est3coloriage(f, G):
                renvoyer vrai();
        fin
        renvoyer faux();
    fin
\end{lstLNat}
    \item On suppose que \code{est3coloriage} est de complexité constante (c'est facilement envisageable). Ainsi, puisqu'il existe $3^n$ coloriages pour un graphe à $n$ sommets, la complexité temporelle de \code{est3colorable} est en $\Theta(3^n)$~: elle est exponentielle en le nombre de sommets.
\end{enumerate} 
\newpage
\section{Exercice 2}
% \begin{enumerate}
    \item \includegraphics[width=0.6\textwidth]{IMG_1076.jpg}
    \item \includegraphics[width=0.6\textwidth]{IMG_1077.jpg}
    \item Comme tous les poids valent 1, le graphe de liaison (qui est un arbre couvrant) donne le plus court chemin de 1 à tous les autres sommets. Ainsi on calcule grâce à la question 2~:
    \begin{align*}
        \code{d = [}&\code{0, 1, 2, 3,}\\
        & \code{2, 3, 4, 4,} \\
        & \code{1, 2, 3, 4]}
    \end{align*}
    Attention, la deuxième ligne concerne les sommets de 9 à 12, pas de 5 à 8 (qui sont décrits par la ligne 3).
    \item \includegraphics[width=0.6\textwidth]{IMG_1077.jpg}
    \item Les graphes sont les mêmes. Le parcours en largeur depuis $s \in V_G$ fournit le plus court chemin pour un graphe $G$ pondéré de façon unitaire.
\end{enumerate}
\newpage
\section{Exercice 3}

\begin{itemize}
    \item 
    \includegraphics[width=0.6\textwidth]{ex3q1.png}
    \item
    Oui, cela est possible selon l'odre de relachement des arêtes. C'est d'ailleurs le cas dans le graphe en exemple aves les sommets 2 et 3. 
    \item \begin{align*}
    \code{d = [}&\code{00, -1, 02, 03,}\\
    \code{02, }& \code{01, 00, 02, 03,}\\
    &\code{02, 02, 00, 01,}\\
    &+\infty\code{, 06, -3, -4]}\\
    \code{parents = [}&\code{1, 5, 2, 3,}\\
    \code{1, }& \code{1, 2, 11, 7,}\\
    &\code{5, 6, 10, 11,}\\
    &\code{13, 16, 16, 12]}
\end{align*}

    \item 
    En un seul relachement dans le bon ordre, on peut avoir tous les sommets bien évalué. En effet, si on prend l'ordre des plus courts chemeins à chaque fois il suffit d'un seul relachement de chaque arêtes pour obtenir les distances minimales.
\end{itemize}

\section{Exercice 4}
\subsection{4.1}

\subsection{4.2}
Pour résoudre cet algorithme, on peut trier les arrètes de du graphes en fonction du poids de leur premier sommet. Ainsi, 
les premières arrètes sont de valeur $-\infty$. On a donc une complexité dépendante de l'algorithme de trie, au mieu $O(nlog(n))$.

\subsection{4.3}
On peut aussi simplement itérer sur les arrètes de $G$.

\subsection{4.4}


\section{Exercice 5}
\begin{itemize}

\item On a $|V_{\psi(G)}| = |V_G| - 1$ et $|E_{\psi(G)}| = |E_G| - 1$, or $\forall x,y \in \mathbb{N}^*, x - y = (x - 1) - (y - 1)$. Donc, $|V_{\psi(G)}|- |E_{\psi(G)}| = |V_G|- |E_G|$.

\item La non-connexité minimal d'un graphe vaut $1 -$ le nombre de composantes connexes. Or, f ne supprime ni n'ajoute de composantes connextes. Donc, la non-connexité minimal est constante.

\item L'acyclicité maximum d'un arbre vaut le nombre de composantes connexes d'un graphe acyclique. D'après l'argument ci-dessus, $\psi$ conserve l'acyclicité maximum.

\item La non acyclicité maximum d'un graphe correspond au nombre maximum d'arrètes a retirer pour que le graphe ne soit plus non acyclique. Or $\psi$ ne supprimant pas de cycle, le nombre de cycles a supprimer reste constant apres l'application de $\psi$

\end{itemize}

\section{Exercice 6}
\begin{itemize}
    \item[$\boxed{\implies}$] Si $G = (V_G, E_G)$ et $H = (V_H, E_H)$ sont isomorphes, alors il existe $\varphi : V_G \to V_H$ une bijection telle que~:
    \begin{equation}
        \label{eq:isomorphism-condition}
        \forall (u_G, v_G) \in E_G,\, \Big(\varphi(u_G),\varphi(v_G)\Big) \in E_H
    \end{equation}
    Notons $\sigma_G : V_G \to \intint{1}{\card{G}}$ la bijection qui a permi de concevoir \code{normaliser(}$G$\code{)}. De même pour $\sigma_H$. On pose~:
    \fonction{\varphi'}{\intint{1}{\card{G}}}{\intint{1}{\card{H}}}{u}{(\sigma_H\circ \varphi \circ \sigma_G^{-1})(u)}
    %cite equation
    De (\ref{eq:isomorphism-condition}), on déduit alors~:
    $$\forall (u_G, v_G) \in E_G,\, \Big((\sigma_H \circ \varphi)(u_G),(\sigma_H \circ \varphi)(v_G)\Big) \in \sigma_H(E_H) = E_{\code{normaliser(}H\code{)}}$$
    Puis car $\sigma_G$ est bijective~:
    $$\forall (u_G, v_G) \in E_{\code{normaliser(}G\code{)}},\, \Big((\sigma_H \circ \varphi \circ \sigma_G^{-1})(u_G),(\sigma_H \circ \varphi \circ \sigma_G^{-1})(v_G)\Big) \in E_{\code{normaliser(}H\code{)}}$$
    Par définition de $\varphi'$ ceci équivaut à~:
    $$ \forall (u_G, v_G) \in E_{\code{normaliser(}G\code{)}},\, \Big(\varphi'(u_G),\varphi'(v_G)\Big) \in E_{\code{normaliser(}H\code{)}}$$
    et traduit exactement le caractère isomorphe de $E_{\code{normaliser(}G\code{)}}$ et $E_{\code{normaliser(}H\code{)}}$.
    \item[$\boxed{\impliedby}$] Si $E_{\code{normaliser(}G\code{)}}$ et $E_{\code{normaliser(}H\code{)}}$ sont isomorphes, alors il existe une bijection~:
    \fonction{\varphi'}{\intint{1}{\card{G}}}{\intint{1}{\card{H}}}{u}{\varphi'(u)}
    telle que~:
    $$ \forall (u_G, v_G) \in E_{\code{normaliser(}G\code{)}},\, \Big(\varphi'(u_G),\varphi'(v_G)\Big) \in E_{\code{normaliser(}H\code{)}}$$
    Notons $\sigma_G : V_G \to \intint{1}{\card{G}}$ la bijection qui a permi de concevoir \code{normaliser(}$G$\code{)}. De même pour $\sigma_H$. On pose~:
    \fonction{\varphi}{V_G}{V_H}{u}{(\sigma_H^{-1}\circ \varphi' \circ \sigma_G)(u)}
    Par définition de $\varphi'$, on déduit alors~:
    $$ \forall (u_G, v_G) \in E_{\code{normaliser(}G\code{)}},\, \Big((\sigma_H^{-1} \circ \varphi' )(u_G),(\sigma_H^{-1} \circ \varphi' )(v_G)\Big) \in \sigma_H^{-1}(E_{\code{normaliser(}H\code{)}}) = E_H$$
    Puis car $\sigma_G$ est bijective~:
    $$ \forall (u_G, v_G) \in E_G,\, \Big((\sigma_H^{-1} \circ \varphi' \circ \sigma_G)(u_G),(\sigma_H^{-1} \circ \varphi' \circ \sigma_G)(v_G)\Big) \in E_H$$
    Par définition de $\varphi$ ceci équivaut à~:
    $$ \forall (u_G, v_G) \in E_G,\, \Big(\varphi(u_G),\varphi(v_G)\Big) \in E_H$$
    et traduit exactement le caractère isomorphe de $G$ et $H$.
\end{itemize}
\section{Exercice 7}
% \begin{lstLNat}
    fonction existeArc(i,j:sommet; G : graphe) : booléen
    début
      retourner appratient(j,G[i]) 
    fin
\end{lstLNat}

\includegraphics[width=0.5\textwidth]{figures/ex7-1.jpg}

\begin{lstLNat}
    fonction ajoutArc(i,j:sommet; G:graphe)
    début
      si non(appratient(i,j,G)) 
        ajouter(j,G[i])
    fin
\end{lstLNat}
    
\includegraphics[width=0.5\textwidth]{figures/ex7-2.jpg}

\begin{lstLNat}
    fonction supprimerArc(i,j:sommet; G:graphe)
    début
      si (appratient(i,j,G))
        supprimer(j,G[i])
    fin
\end{lstLNat}

\includegraphics[width=0.5\textwidth]{figures/ex7-3.jpg}




\end{document}



