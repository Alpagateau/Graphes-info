Soit un graphe $G$ non orienté. Soit $x$ un sommet de $G$ de degré 1 et $e$ son unique arête.
$x$ appartient à une composante connexe de $G$ qui est composé d'au moins deux sommet ca $x$ est de degré 1. Donc en retirant le sommet $x$, le sous graphe obtenu a le même nombre de composante connexe, celles différentes de $X$ sont inchangé et $X$  est toujours une composante connexe avec au moins un sommet.
Donc si $G$ est connexe alors le sous graphe de $G$ est connexe, sinon il reste non connexe.

\paragraph{}

Pout tout cycle, chaque sommet est de degré au moins 2. Or $x$ est de degré 1 donc il ne peut pas appartenir à un cycle. 
Donc si $G$ est cyclique, alors il existe un cycle ne passant pas par $x$. Donc le sous-graphe obtenu en retirant $x$ conserve ce cycle. D'où la conservation de la non acyclicité par $\Psi$.

De plus, retirer un sommet et une arête ne peut pas créer de cycle donc $\Psi$ conserve aussi l'acyclicité.


