\begin{itemize}

\item On a $|V_{\psi(G)}| = |V_G| - 1$ et $|E_{\psi(G)}| = |E_G| - 1$, or $\forall x,y \in \mathbb{N}^*, x - y = (x - 1) - (y - 1)$. Donc, $|V_{\psi(G)}|- |E_{\psi(G)}| = |V_G|- |E_G|$.

\item La non-connexité minimal d'un graphe vaut $1 -$ le nombre de composantes connexes. Or, f ne supprime ni n'ajoute de composantes connextes. Donc, la non-connexité minimal est constante.

\item L'acyclicité maximum d'un arbre vaut le nombre de composantes connexes d'un graphe acyclique. D'après l'argument ci-dessus, $\psi$ conserve l'acyclicité maximum.

\item La non acyclicité maximum d'un graphe correspond au nombre maximum d'arrètes a retirer pour que le graphe ne soit plus non acyclique. Or $\psi$ ne supprimant pas de cycle, le nombre de cycles a supprimer reste constant apres l'application de $\psi$

\end{itemize}
