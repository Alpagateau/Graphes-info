On associe les couleurs de la façon suivante~:
\begin{itemize}
    \item Rouge : salle déjà visitée
    \item Blanc : couloir jamais emprunté
    \item Bleu : couloir emprunté une fois : on peut remonter par là
    \item Vert : couloir emprunté deux fois : on a fait l'allez-retour.
\end{itemize}

\begin{lstLNat}
    algo quitter_labyrinthe
    tantque non(estSalleSortie()) faire
    debut
        si non(estSalleRouge()) alors
        debut
            marquerSalleRouge()
        fin
        si existeCouloirBlanc() alors
        debut
            accederPorteBlanche()
            traverserCouloir()
            marquerCouloirBleu()
        fin
        sinon
        debut
            si existeCouloirBleu() alors
            debut
                accederPorteBleue()
                traverserCouloir()
                marquerCouloirVert()
            fin
        fin
    fin

\end{lstLNat}

Cet algorithme parcourt au plus deux fois chaque couloir. L'algorithme est donc de complexité temporelle linéaire en le nombre de couloirs. Les tests ont été effectués sur les labyrinthes suivants~:

\begin{itemize}
    \item Grille $5\times 5$, Thésée explore toute la grille, il finit par trouver la sortie.
    \item Graphe complet de 5 sommets, Thésée explore tous les couloirs, il finit par trouver la sortie
\end{itemize}

De façon générale les cycles sont prevenus par le marquage des salles et couloirs déjà visités, ce qui garantit que l'algorithme finit par trouver la sortie un en temps fini et moindre.