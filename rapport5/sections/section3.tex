\begin{enumerate}
    \item Lorsque l'arc $e_{i-1}$ est visité, le sommet $s_i$ est encore gris, car il existe au moins un chemin partant de celui-ci vers un sommet blanc ou gris.
    \item lorsque l'arc $(c,a)$ est visité, le sommet $c$ est encore gris, car il existe au moins un chemin vers un sommet blanc ou gris partant de celui-ci.
    \centering\includegraphics[width=0.5\textwidth]{IMG_E1071.JPG}
    \item \begin{lstLNat}
    fonction parcoursProfondeur(G : graphe)
    debut
        n <- dernierSommet(G) ; 
        C <- allouerTableau(n,BLANC);

        pour s de 1 à n faire 
            si C[s]=BLANC alors
                visiter(s,G,C);
        retourner vrai()
    fin

    fonction visiter(s : sommet , G : graphe , C : tableau de couleurs)
    debut
        C[s] <- GRIS ;
        L <- listeArcsSortants(s,G) ; 
        k <- longueur(L) ; 

        pour i de 1 à k faire
            debut
            t <- 2ndeExtrémité(iemeArc(i,L)) ;
            si C[t] = BLANC alors
                visiter(t,G,C)
            sinon si C[t] = GRIS alors
                retourner faux()
        C[s] <- NOIR
    fin
    \end{lstLNat}
\end{enumerate}