\begin{enumerate}
    \item \includegraphics[width=0.6\textwidth]{IMG_1076.jpg}
    \item \includegraphics[width=0.6\textwidth]{IMG_1077.jpg}
    \item Comme tous les poids valent 1, le graphe de liaison (qui est un arbre couvrant) donne le plus court chemin de 1 à tous les autres sommets. Ainsi on calcule grâce à la question 2~:
    \begin{align*}
        \code{d = [}&\code{0, 1, 2, 3,}\\
        & \code{2, 3, 4, 4,} \\
        & \code{1, 2, 3, 4]}
    \end{align*}
    Attention, la deuxième ligne concerne les sommets de 9 à 12, pas de 5 à 8 (qui sont décrits par la ligne 3).
    \item \includegraphics[width=0.6\textwidth]{IMG_1077.jpg}
    \item Les graphes sont les mêmes. Le parcours en largeur depuis $s \in V_G$ fournit le plus court chemin pour un graphe $G$ pondéré de façon unitaire.
\end{enumerate}