
\begin{itemize}
    \item 
    \includegraphics[width=0.6\textwidth]{ex3q1.png}
    \item
    Oui, cela est possible selon l'odre de relachement des arêtes. C'est d'ailleurs le cas dans le graphe en exemple aves les sommets 2 et 3. 
    \item \begin{align*}
    \code{d = [}&\code{00, -1, 02, 03,}\\
    \code{02, }& \code{01, 00, 02, 03,}\\
    &\code{02, 02, 00, 01,}\\
    &+\infty\code{, 06, -3, -4]}\\
    \code{parents = [}&\code{1, 5, 2, 3,}\\
    \code{1, }& \code{1, 2, 11, 7,}\\
    &\code{5, 6, 10, 11,}\\
    &\code{13, 16, 16, 12]}
\end{align*}

    \item 
    En un seul relachement dans le bon ordre, on peut avoir tous les sommets bien évalué. En effet, si on prend l'ordre des plus courts chemeins à chaque fois il suffit d'un seul relachement de chaque arêtes pour obtenir les distances minimales.
\end{itemize}
